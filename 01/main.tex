\documentclass{article}
\usepackage{tabularx}
\usepackage{geometry}
\usepackage{multicol}
\usepackage{array}
\usepackage{mathtools}
\usepackage{tabularx}
\usepackage{hhline}
\usepackage{amsmath}
\usepackage{amssymb}
\usepackage{titlesec}
\usepackage{enumerate}

\geometry{
    top=25mm,
    bottom=25mm,
    left=25mm,
    right=25mm
}


\renewcommand\tabularxcolumn[1]{m{#1}}
\renewcommand{\arraystretch}{1.4}



\begin{document}


\section*{1. úloha}

\noindent \textbf{Definice 1.0} Charakteristickou funkci $\mathnormal{\chi}_A$ množiny $A$ zavedeme
\[
    \chi_A(x) = 
    \begin{dcases}
        \,1 & \text{iff}\quad x \in A\\
        \,0 & \text{iff}\quad x \in A^{\mathsf{c}} \\
    \end{dcases}
    .
\]


\vspace{0.5em}
\noindent \textbf{Zadání 1.0} Nechť $s_1, s_2: \mathbb{R} \to \mathbb{C}$ jsou zadané předpisy
\begin{align*}
    s_1(t) &\coloneqq A \chi_{I}(t), \\
    s_2(t) &\coloneqq A \chi_{I}(t) \cos\left(\frac{2\pi}{T_1}t\right),
\end{align*}
pro $I = \left[-\frac{T_1}{4},\frac{T_1}{4}\right]$, $A \in \mathbb{C}$ a $T_1\in \mathbb{R^{+}}$. Spočtěte  $\mathcal{R}[s_1, s_2](\tau)$.
\vspace{0.5em}

\noindent \textbf{Řešení}
\begin{align*}
    \mathcal{R}[s_1, s_2](\tau) = \langle s_1(t + \tau), s_2(t)\rangle_{\mathcal{H}} = \int_{\mathbb{R}} s_1(t+\tau)s_2^{*}(t)\,\mathrm{d}t.
\end{align*}
Nyní zkonstruujeme disjunktní dělení množiny $\mathbb{R}$
\[
    \mathfrak{D} = \{I_1, I_2, I_3, I_4\},
\]
kde $I_1 = \left]-\infty, \frac{T_1}{2}\right]$, $I_2 = \left]-\frac{T_1}{2}, 0\right]$, $I_3 = \left]0, \frac{T_1}{2}\right]$ a $I_4 = \left]\frac{T_1}{2}, \infty\right[$. Autokorelační funkci můžeme rozepsat
\begin{align*}
    \mathcal{R}[s_1, s_2](\tau) &= \left(\sum_{I \in \mathfrak{D}} \chi_{I}(\tau) \int_{I}\right)s_1(t + \tau)s_2^{*}(t) \,\mathrm{d}t = \\ &= \chi_{I_1}(\tau) \cdot 0 + \chi_{I_2}(\tau) \int_{I_2} A A^{*} \cos\left(\frac{2 \pi}{T_1} t\right)\,\mathrm{d}t + \chi_{I_3}(\tau) \int_{I_3} A A^{*} \cos\left(\frac{2 \pi}{T_1} t\right)\,\mathrm{d}t + \chi_{I_4}(\tau) \cdot 0 = \\ &= \chi_{I_2}(\tau) \frac{|A|^2 T_1}{\pi} \cos ^2\left(\frac{\pi }{T_1}\tau\right) + \chi_{I_3}(\tau) \frac{|A|^2 T_1}{\pi} \cos ^2\left(\frac{\pi}{T_1} \tau \right) = \\ &= \chi_{\left[\frac{-T_1}{2}, \frac{T_1}{2}\right]}(\tau) \frac{|A|^2 T_1}{\pi} \cos^2\left(\frac{\pi}{T_1}\tau \right).
\end{align*}

\vspace{0.5em}
\noindent \textbf{Zadání 1.1} Spočtěte energie $s_1$ a $s_2$ ze zadání 1.0 a také jejich vzájemnou energii.

\vspace{0.5em}
\noindent\textbf{Řešení}
\[
    E[s_1] = \langle s_1, s_1 \rangle_{\mathcal{H}} = \int_{\mathbb{R}}s_1(t) s_1^{*}(t)\,\mathrm{d}t = \int_{-\frac{T_1}{4}}^{\frac{T_1}{4}} |A|^2 \,\mathrm{d}t = \frac{|A|^2T_1}{2}.
\]
\[
    E[s_2] = \langle s_2, s_2 \rangle_{\mathcal{H}} = \int_{\mathbb{R}}s_2(t) s_2^{*}(t)\,\mathrm{d}t = \int_{-\frac{T_1}{4}}^{\frac{T_1}{4}} |A|^2 \cos^2\left(\frac{2\pi}{T_1} t \right) \,\mathrm{d}t = \frac{|A|^2}{2} \int_{-\frac{T_1}{4}}^{\frac{T_1}{4}} 1 + \cos\left(\frac{4\pi}{T_1}t \right)\,\mathrm{d}t = \frac{|A|^2T_1}{4}.
\]
\[
    E[s_1, s_2] = \langle s_1, s_2 \rangle_{\mathcal{H}} = \int_{\mathbb{R}}s_1(t) s_2^{*}(t)\,\mathrm{d}t =|A|^2 \int_{-\frac{T_1}{4}}^{\frac{T_1}{4}} \cos\left( \frac{2\pi}{T_1}t\right)\,\mathrm{d}t = \frac{|A|^2T_1}{\pi}.
\]

\newpage

\section*{2. úloha}

\noindent \textbf{Věta 2.0} Mějme periodickou funkci $f$ se základní periodou $T_0$ s předpisem
\[
    f(t) = \sum_{\substack{n \in A \\ t \in M}} K_n \chi_{M_n}(t),
\]
kde $K_n \in \mathbb{R},\,M_n \subset \mathbb{R} \,\forall n \in A \subset \mathbb{Z}$ a $M = \bigcup_{n \in A} M_n$. Navíc každá $M_n$ je souvislá, tedy je intervalem $[a_n,b_n)$. Potom Fourierovy koeficienty funkce $f$ budou
\[
    c_m = \frac{1}{T_0}\sum_{n \in A} 2 K_n \nu_n e^{-i \omega_0 m \xi_n} \text{sinc}(\omega_0 m \nu_n),
\]
kde $T_0 = \sup_{n\in A}(b_n) - \inf_{n\in A}(a_n),\,\omega_0 = \frac{2\pi}{T_0}$
\[
    \nu_n \coloneqq \frac{b_n-a_n}{2}\quad \text{a}\quad \xi_n \coloneqq \frac{a_n+b_n}{2}.
\]
\noindent \textbf{Důkaz}
Koeficienty určíme z
\begin{align*}
    c_m = \langle f, e^{i\omega_0 m}\rangle_\mathcal{H} = \frac{1}{T_0} \int_{M} f(t)e^{-i \omega_0 m t}\,\mathrm{d}t =
\frac{1}{T_0} \int_{M} \sum_{\substack{n \in A \\ t \in M}} K_n \chi_{M_n}(t) e^{-i \omega_0 m t}\,\mathrm{d}t,
\end{align*}
zaměníme pořadí integrace a sumace (odvoláváme se na Fubiniho větu)
\begin{align*}
    c_m =
\frac{1}{T_0} \sum_{\substack{n \in A \\ t \in M}}\int_{M_n} K_n e^{-i \omega_0 m t}\,\mathrm{d}t = \frac{1}{T_0} \sum_{\substack{n \in A \\ t \in M}} \frac{K_n \left(e^{-i m \omega _0 b_n}-e^{ -im \omega _0 a_n}\right)}{-i m \omega _0}.
\end{align*}
Dále zavedeme pomocné $\nu_n$ a $\xi_n$ a upravíme
\begin{align*}
    c_m = \frac{1}{T_0} \sum_{\substack{n \in A \\ t \in M}} \frac{K_n \left(e^{-i m \omega _0 \left(\xi _n-\nu _n\right)}-e^{-i m \omega _0 \left(\nu _n+\xi _n\right)}\right)}{i m \omega _0} = \frac{1}{T_0}\sum_{n \in A} 2 K_n \nu_n e^{-i \omega_0 m \xi_n} \text{sinc}(\omega_0 m \nu_n).
\end{align*}
\flushright{$\square$}


\vspace{0.5em}
\noindent \flushleft{\textbf{Zadání 2.0}} Více informací k zadání je v úkolu.
\begin{enumerate}[(i)]
    \item Určete základní periody signálů $s_3$ a $s_4$.
    \item Určete předpisy pro signály $s_3$ a $s_4$ na jedné jejich (vhodně zvolené) základní periodě.
    \item Určete koeficienty Fourierovy řady $c_{4,n}$ signálu $s_4$ a koeficienty Fourierovy řady $r_{4,n}$ korelační funkce $\mathcal{R}[s_4](\tau)$.
    \item Určete stejnosměrné složky $\text{DC}\left[s_3\right]$ , $\text{DC}\left[s_4\right]$ signálů $s_3$, resp. $s_4$
    \item Ověřte Parsevalovu větu pro signál $s_3$, tj. určete výkon $P[s_3]$ v čase a pak z koeficientů $c_{3,n}$ – měli byste dostat stejnou hodnotu.
    \item Určete výkon $P[s_4]$ a kolik procent výkonu je neseno harmonickou složkou s indexem $n = 1$ obecně a pro $\beta = \frac{1}{4}$.
\end{enumerate}

\newpage

\noindent \textbf{Řešení}
\begin{enumerate}[(i)]
\item Základní perioda $s_3$ a $s_4$ je $T_2$.
\item
\[
    s_3(t) = A\chi_{[0,T_3]}(t) -A \chi_{[T_3,T_2]}(t),\,\forall t\in [0, T_2]
\]
a
\[
    s_4(t) = \frac{2A}{T_2},\,\forall t\in \left[-\frac{T_2}{2}, \frac{T_2}{2}\right]
\]

\item K řešení využijeme věty 2.0, přičemž jednotlivé koeficienty jsou
\[
    T_0 = T_2,\,K_1 = A,\,K_2=-A,\,\nu_1 = \frac{T_3}{2}, \,\nu_2=\frac{T_2-T_3}{2},\, \xi_1 = \frac{T_3}{2},\,\xi_2=\frac{T_3+T_2}{2},\, \text{a }\,\omega_0 = \frac{2\pi}{T_2},
\]

potom
\begin{align*}
    c_{4,n} &= \frac{A T_3}{T_2} e^{-\frac{i \pi  n T_3}{T_2}} \text{sinc}\left(\frac{\pi  n T_3}{T_2}\right)-A \frac{T_2-T_3}{T_2} e^{-\frac{i \pi  n \left(T_2+T_3\right)}{T_2}} \text{sinc}\left(\frac{\pi  n \left(T_2-T_3\right)}{T_2}\right) = \\ &= A \beta e^{-i \pi  n \beta} \text{sinc}\left(\pi  n \beta\right)-A \left(1-\beta\right) e^{-i \pi  n \left(1+\beta\right)} \text{sinc}\left(\pi  n \left(1-\beta\right)\right) = \\ &= \frac{i A}{\pi n} \left(e^{-2 i \pi  \beta  n}-1\right)
\end{align*}
Dále určíme Fourierovy koeficienty autokorelační funkce
\begin{multline*}
    r_{4, n} = \frac{1}{T_2}\int_0^{T_3} |A|^2 \left(1-4\frac{\tau}{T_2}\right)e^{-i\omega_0 n \tau}\,\mathrm{d}\tau + \\ + \frac{1}{T_2}\int_{T_3}^{T_2-T_3} |A|^2 \left(1-4\beta\right)e^{-i\omega_0 n \tau}\,\mathrm{d}\tau + \frac{1}{T_2}\int_{T_2-T_3}^{T_2} |A|^2 \left(-3+4\frac{\tau}{T_2}\right)e^{-i\omega_0 n \tau}\,\mathrm{d}\tau
\end{multline*}
Nyní manipulací s integračními mezemi převedeme na
\begin{align*}
    r_{4,n}=\frac{1}{T_2}\int_{-T_3}^{T_3} |A|^2 \left(1-4\frac{|\tau|}{T_2}\right)e^{-i\omega_0 n \tau}\,\mathrm{d}\tau +\frac{1}{T_2}\int_{T_3}^{T_2-T_3} |A|^2 \left(1-4\beta\right)e^{-i\omega_0 n \tau}\,\mathrm{d}\tau,
\end{align*}
koplexní exponenciálu přepíšeme do trigonometrickéhi tvaru
\begin{align*}
    r_{4,n}=\frac{1}{T_2}\int_{-T_3}^{T_3} |A|^2 \left(1-4\frac{|\tau|}{T_2}\right)\left(\cos(\omega_0 n \tau) + i \sin(\omega_0 n \tau)\right)\,\mathrm{d}\tau +\frac{1}{T_2}\int_{T_3}^{T_2-T_3} |A|^2 \left(1-4\beta\right)e^{-i\omega_0 n \tau}\,\mathrm{d}\tau
\end{align*}
a využijeme poznatků o integraci přes sudé/liché funkce na symetrickém intervalu, získáme tak
\begin{align*}
    r_{4,n}=\frac{2}{T_2}\int_{0}^{T_3} |A|^2 \left(1-4\frac{\tau}{T_2}\right)\cos(\omega_0 n \tau)\,\mathrm{d}\tau +\frac{1}{T_2}\int_{T_3}^{T_2-T_3} |A|^2 \left(1-4\beta\right)e^{-i\omega_0 n \tau}\,\mathrm{d}\tau.
\end{align*}
První integrál je roven
\begin{align*}
    \frac{|A|^2}{\pi^2 n^2 T_2} \left(\pi  n \left(T_2-4 T_3\right) \sin \left(\frac{2 \pi  n T_3}{T_2}\right)-2 T_2 \cos \left(\frac{2 \pi  n T_3}{T_2}\right)+2 T_2\right) = \\ = 2|A|^2 \left(\frac{2 \sin ^2\left(\frac{\pi  n T_3}{T_2}\right)}{\pi ^2 n^2}+\frac{T_3 \left(T_2-4 T_3\right) \text{sinc}\left(\frac{2 \pi  n T_3}{T_2}\right)}{T_2^2}\right) = \\ = 2|A|^2 \left(2\beta^2\text{sinc}^2(\pi n\beta) + \beta (1-4\beta)\text{sinc}(2\pi n \beta)\right)
\end{align*}
a druhý integrál je roven, za použití věty 2.0 ($T_0 = T_2$, $K_1 = |A|^2(1-4\beta)$, $\nu_1 = \frac{T_2 - 2T_3}{2}$, $\xi_1 = \frac{T_2}{2}$ a $\omega_0 = \frac{2 \pi}{T_2}$)
\[
    \frac{|A|^2}{T_2} (1-4 \beta ) e^{-i \pi  n} \left(T_2-2 T_3\right) \text{sinc}\left(\frac{\pi  n \left(T_2-2 T_3\right)}{T_2}\right) = |A|^2 (1-4 \beta ) e^{-i \pi  n} \left(1-2 \beta\right) \text{sinc}\left(\pi  n \left(1-2\beta \right)\right).
\]
\newpage
Celkově tedy dostáváme
\begin{align*}
    r_{4,n} &= 2|A|^2 \left(2\beta^2\text{sinc}^2(\pi n\beta) + \beta (1-4\beta)\text{sinc}(2\pi n \beta)\right) +|A|^2 (1-4 \beta ) e^{-i \pi  n} \left(1-2 \beta\right) \text{sinc}\left(\pi  n \left(1-2\beta \right)\right) = \\ &= -\frac{A^2 \left(1-e^{-2 i \pi  \beta  n}\right)^2}{\pi ^2 n^2},
\end{align*}
což je totéž co $|c_{4,n}|^2$.
\item Konstantní komponenty signálu $s_3$ je $0$, jelikož je lichý. $\text{DC}\left[s_4\right]$ určíme následovně
\[
    \text{DC}\left[s_4 \right] = \langle s_3 \rangle = \frac{1}{T_2} \int_{0}^{T_2} s_3(t)\,\mathrm{d}t = \frac{1}{T_2}\left(A\int_{0}^{T_3}\,\mathrm{d}t -A\int_{T_3}^{T_2}\,\mathrm{d}t \right) = A(2\beta-1).
\]
\item Máme ověřit, že
\[
    \lVert s_3\rVert_{L^2}^2 = \lVert \left\{ c_{3, n}\right\}_{n\in \mathbb{Z}}\rVert_{\ell^2}^2.
\]
Na levé straně máme
\[
   \lVert s_3\rVert_{L^2}^2= \frac{1}{T_2}\int_{-\frac{T_2}{2}}^{\frac{T_2}{2}} \frac{4|A|^2}{T_2^2} t^2\,\mathrm{d}t = \frac{|A|^2}{3}
\]
a na pravé
\[
    \lVert \left\{ c_{3, n}\right\}_{n\in \mathbb{Z}}\rVert_{\ell^2}^2 = \sum_{n \in \mathbb{Z}} |c_{3,n}|^2 = \sum_{n \in \mathbb{Z} \setminus \{0\}} \left|\frac{iA (-1)^n}{n \pi}\right|^2 = 2\sum_{n=1}^{\infty} \frac{|A|^2}{n^2 \pi^2} = \frac{2}{\pi^2}\sum_{n=1}^{\infty}\frac{1}{n^2} = \frac{|A|^2}{3}.
\]

\item Nejprve určíme
\[
    \text{P}\left[s_4\right] = \langle |s_4|^2\rangle = \frac{1}{T_2}\int_{0}^{T_2}|s_4|^2 \,\mathrm{d}t = \frac{1}{T_2}\left(|A|^2\int_{0}^{T_3}\,\mathrm{d}t +|A|^2\int_{T_3}^{T_2}\,\mathrm{d}t \right) = |A|^2,
\]
dále určíme výkon harmonické složky pro $n = 1$, tedy
\[
    |c_{4,1}|^2 = \left| \frac{i A \left(e^{-2 i \pi  \beta-1 }\right)}{\pi }\right|^2 = \frac{4 |A|^2 \sin ^2(\pi  \beta )}{\pi ^2} = 4|A|^2\beta^2\text{sinc}^2(\pi\beta),
\]
a poměr výkonu
\[
    \eta = \frac{|c_{4,1}|^2}{\text{P}\left[s_4\right]} = 4\beta^2 \text{sinc}^2(\pi\beta).
\]
Pro $\beta = \frac{1}{4}$ je $\eta = \frac{2}{\pi}$.

\end{enumerate}

\end{document}
